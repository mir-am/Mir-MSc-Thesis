% Persian Abstarct

\thispagestyle{plain}
\pagenumbering{arabic}

% Adding abstract to TOC
\addcontentsline{toc}{section}{چکیده}

\noindent
\textbf{\Large چکیده}
\vskip 1 cm
\noindent در دهه اخیر، یادگیری ماشین برای حل کردن مسائل با الگوهای پیچیده استفاده شده است. دسته‌بندی یکی از روش‌های اصلی یادگیری است که مسائلی نظیر تشخیص چهره، تشخیص متون و تشخیص بیماری‌ها را حل می‌کند. ماشین بردار پشتیبان  (\lr{SVM}) یکی از روش‌های شناخته شده  دسته‌بندی است که دقت و تعمیم‌پذیری خوبی دارد. دسته‌بندهای مختلفی بر پایه \lr{SVM} در سال‌های اخیر ارائه شده است. در میان آن‌ها ماشین بردار پشتیبان دو قلو (\lr{TSVM}) بیشتر مورد توجه بوده است.

ایده اصلی روش \lr{TSVM} پیدا کردن دوابرصفحه غیرموازی برای دسته‌بندی داده‌ها است. بطوریکه دو مسئله بهینه‌سازی دوگان با اندازه کوچک‌تر از مسئله دوگان در \lr{SVM} حل می‌شود. از این رو دسته‌بند \lr{TSVM} در تئوری ۴ برابر سریع‌تر از \lr{SVM} است. با وجود اینکه دسته‌بند \lr{TSVM} دقت و مرتبه زمانی بهتری نسبت به \lr{SVM} دارد، نقاط ضعفی ما نند حساسیت به نمونه‌های پرت و نویزی، برازش بیش از حد (\lr{Overfitting}) و پیچیدگی محاسباتی بالا برای مجموعه داده‌های بزرگ را دارد.  در این پژوهش با هدف برطرف کردن نقطه ضعف بیان شده در \lr{TSVM}، دو دسته‌بند ماشین بردار پشتیبان دوقلو کمترین مربعات مبتنی بر نزدیک‌ترین همسایه (\lr{KNN-LSTSVM}) و ماشین بردار پشتیبان دوقلو مبتنی بر رگولارسیون و نزدیک‌ترین همسایه (\lr{RKNN-TSVM}) ارائه می‌شود.  

دسته‌بند پیشنهادی (\lr{KNN-LSTSVM}) با ساخت گراف درون کلاسی و برون کلاسی به نمونه‌های آموزشی وزن می‌دهد و همچنین نمونه‌های حاشیه‌ای هر کلاس را مشخص می‌کند. این مشخصه دسته‌بند پیشنهادی را نسبت به نمونه‌های نویزی و پرت مقاوم‌تر از روش  \lr{TSVM} می‌کند. همچنین با بکارگیری روش کمترین مربعات، مدل خروجی در دسته‌بند \lr{KNN-LSTSVM} با حل کردن دو دستگاه معادلات خطی بدست می‌آید. بطوریکه سرعت یادگیری این دسته‌بند به طور قابل توجه‌ای افزایش یافته است. دسته‌بند \lr{RKNN-TSVM} به نمونه‌ها براساس فاصله بین نزدیک‌ترین همسایه‌هایشان وزن می‌دهد و ریسک ساختاری را در مسئله بهینه‌سازی کمینه می‌کند.

روش‌های پیشنهادی بر روی مجموعه داده‌های مصنوعی و واقعی ارزیابی و بررسی شده است. نتایج ارزیابی نشان می‌دهد که دسته‌بند های پیشنهادی \lr{KNN-LSTSVM} و \lr{RKNN-TSVM} نسبت به سایر دسته‌بندهای مشابه از نظر دقت و سرعت یادگیری بهتر عمل کرده‌اند.

\vskip 2cm
\noindent
\textbf{واژه‌های کلیدی:}
