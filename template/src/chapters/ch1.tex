% This adds a rule bar and sections title to header.
\pagestyle{fancy}
\fancyhf{}
\cfoot{\thepage}
\fancyhead[R]{\rightmark}

\chapter{کلیات} \label{ch:1}
\section{مقدمه} \label{sec:1:1}
\gls*{ML} به عنوان یکی از شاخه‌های پرکاربرد هوش مصنوعی در چند دهه اخیر بسیار مورد توجه دانشگاه‌ها و صنعت بوده است \cite{jordan2015}. یادگیری ماشین به دنبال شناسایی خودکار الگوهای معنادار از داده‌ها است. در سال‌های اخیر، فناوری‌های مبتنی بر یادگیری ماشین گسترش و توسعه یافته‌اند. به طور مثال، نتایج موتورهای جستجو با روش‌های یادگیری ماشین بهبود یافته است. سامانه‌های تشخیص تقلب در تراکنش‌های بانکی و تشخیص چهره در دوربین‌های دیجیتال از دیگر کاربردهای یادگیری ماشین هستند.

الگوها در مسائل اشاره شده مانند تشخیص چهره بسیار پیچیده هستند. بطوریکه یک برنامه نویس نمی‌تواند با دستورالعمل‌های صریح الگوهای پیچیده را تشخیص دهد. با این حال روش‌های یادگیری ماشین با داشتن داده‌های فراوان می‌توانند الگوهای پیچیده را شناسایی کنند. مزیت اصلی روش‌های یادگیری ماشین سازگاری با تغییرات در محیط است \cite{shalev2014}. به عنوان مثال، اعداد دست‌نوشته برای هر فرد دارای الگوهای متفاوت می‌باشد.

به طور کلی روش‌های یادگیری ماشین به دو دسته \gls{SupV} و \gls{UnSupV} تقسیم می‌شوند \cite{shalev2014}. روش‌های بانظارت یا \gls{cls} توسط نمونه‌های آموزشی با برچسب آموزش داده می‌شوند. برای مثال، در تشخیص خودکار بیماری قلب، نمونه‌های آموزشی شامل ویژگی‌ها و علائم بیمار است و برچسب‌ها نیز نشان می‌دهد که آیا یک شخص بیماری قلبی دارد یا خیر. بعد از اتمام فرآیند یادگیری با نمونه‌های آموزشی، روش بانظارت باید نمونه‌های بدون برچسب را تشخیص دهد. از طرف دیگر،نمونه‌های آموزشی در یادگیری بدون نظارت فاقد برچسب هستند. روش‌های بی‌نظارت یا \gls{clus} معمولا نمونه‌ها را به گروه‌های مشابه تقسیم می‌کنند.

روش‌های دسته‌بندی در بسیاری از مسائل مختلف استفاده شده‌اند. زیرا مسائل زیادی به صورت ویژگی‌ها و متغیر هدف بیان می‌شوند. برای مثال، می‌توان به مسائل تشخیص چهره، تشخیص \gls{spam}، تشخیص بیماری‌ها، دسته‌بندی متن و تشخیص حمله به شبکه‌های کامپیوتری اشاره کرد.   به طور کلی الگوریتم‌های دسته‌بندی دارای دو مرحله هستند:
\begin{enumerate}
	\item مرحله آموزش: در این مرحله، یک مدل خروجی از داده‌های آموزشی ساخته می‌شود.
	\item مرحله تست: در این مرحله، مدل ساخته شده برای پیش‌بینی برچسب یک نمونه تست استفاده می‌شود.
\end{enumerate}

الگوریتم‌های بسیار زیادی برای دسته‌بندی داده‌ها توسعه و گسترش یافته‌اند. برخی از مشهورترین آن‌ها شامل \gls{DT}، \gls{NNG}گراف ، \gls{Bayes}، \gls{ANN} و ماشین بردار پشتیبان( \gls{SVM}) هستند \cite{kotsiantis2007}. هر کدام از این الگوریتم‌های دسته‌بندی نقاط قوت و ضعفی دارند و برای مسائل مشخصی بهتر عمل می‌کنند. در میان روش‌های دسته‌بندی اشاره شده، ماشین بردار پشتیبان دقت و تعمیم‌پذیری بهتری دارد. این روش دسته‌بندی توسط وپنیک\LTRfootnote{Vapnik} و کورتس\LTRfootnote{Cortes} در سال 1995 ارائه شد \cite{vapnik1995}. 

ماشین بردار پشتیبان بر پایه کمینه کردن \gls{SRisk} طراحی شده است \cite{vapnik1998}. ایده اصلی \gls{SVM} پیدا کردن یک \gls{Hplane} با بیشترین فاصله ممکن از داده‌های دو کلاس می‌باشد. بطوریکه یک مسئله  \gls{Opt} از نوع برنامه‌ریزی درجه دو (\gls{QPP}) برای بدست آوردن چنین ابرصفحه‌ای حل می‌شود این روش یادگیری در مسائل مختلف مانند تشخیص آریتمی‌های قلبی \cite{nasiri2009}، شناسایی نفوذ به شبکه‌های کامپیوتری \cite{raman2017}، دسته‌بندی متن\cite{lee2012} و شناسایی \gls{spam} \cite{zoubi2018} مورد استفاده قرار گرفته است.

در دو دهه گذشته، پژوهشگران دسته‌بندهایی مبتنی بر روش ماشین بردار پشتیبان ارائه کرده‌اند \cite{nayak2015}. در میان گسترش‌های روش \gls*{SVM}، ماشین بردار پشتیبان دو قلو (\gls{TSVM}) بیشتر از سایرین مورد توجه پژوهشگران قرار گرفته است. روش \lr{TSVM} با هدف بهبود پیچیدگی زمانی \lr{SVM}  در سال 2007 ارائه گردید \cite{jayadeva2007}. ایده اصلی این روش یادگیری، بدست آوردن دو ابرصفحه غیر موازی است. بطوریکه هر ابرصفحه غیر موازی به نمونه‌های کلاس خود نزدیک است و نمونه‌های کلاس مقابل دور می‌شود. دو مسئله بهینه‌سازی کوچک از نوع برنامه‌ریزی درجه دو برای بدست آوردن این دو ابرصفحه غیر موازی حل می‌گردد. در حالی‌که در روش \lr{SVM} یک مسئله بهینه‌سازی بزرگ حل می‌شود. در نتیجه، روش ماشین بردار پشتیبان دو قلو در تئوری 4 برابر سریع‌تر از روش \lr{SVM} است.

%در سال 2001، ماشین بردار پشتیبان مبتنی بر مفهوم نزدیکی\footnote{\lr{Proximal Support Vector Machine (PSVM)}}  (\lr{PSVM}) ارائه شد \cite{mang2001}. در این روش دو ابرصفحه موازی برای دسته‌بندی نمونه‌ها ایجاد می‌شود. در سال 2002، ماشین بردار پشتیبان فازی\footnote{\lr{Fuzzy Support Vector Machine (FSVM)}}  (\lr{FSVM}) \cite{lin2002} ارائه گردید. در این روش به هر یک از نمونه‌های هر دو کلاس، تعلق فازی داده می‌شود. بطوریکه اثر نمونه‌های نویزی و پرت در ایجاد مدل خروجی کم خواهد شد. در سال 2006، ماشین بردار پشتیبان با رویکرد مقدار ویژه تعمیم یافته\footnote{\lr{Generalized Eigenvalue Proximal Support Vector Machine (GEPSVM)}}  (\lr{GEPSVM}) ارائه شد \cite{mang2006}. برخلاف روش \lr{PSVM}، این روش دو ابرصفحه غیر موازی ایجاد می‌کند که هر یک از این ابرصفحه‌ها به نمونه‌های کلاس خود نزدیک است و از نمونه‌های کلاس مقابل تا جای ممکن فاصله می‌گیرد. همچنین روش \lr{PSVM} بر روی مسئله \lr{XOR} عملکرد بهتری نسبت به روش \lr{SVM} اصلی دارد.



%در دهه اخیر، دسته‌بندهای مختلفی بر مبنای روش ماشین بردار پشتیبان دو قلو ارائه شده است \cite{ding2017,huang2018}. در سال 2009، ماشین بردار پشتیبان دو قلو کمترین مربعات\footnote{\lr{Least Squares Twin Support Vector Machine (LS-TSVM)}}  (\lr{LS-TSVM}) معرفی شد \cite{kumar2009}. در این روش، دو دستگاه معادلات خطی به جای دو مسئله بهینه‌سازی درجه دو حل می‌شود. در نتیجه، سرعت یادگیری روش \lr{LS-TSVM} به طور قابل توجه‌ای بیشتر از روش \lr{TSVM} اصلی بر روی مجموعه داده‌های بزرگ می‌باشد. در سال 2012، ماشین بردار پشتیبان دو قلو وزن دار با اطلاعات محلی\footnote{\lr{Weighted Twin Support Vector Machine with Local Information (WLTSVM)}}  (\lr{WLTSVM}) ارائه شد \cite{ye2012}. برخلاف \lr{TSVM} اصلی، این روش شباهت و اهمیت نمونه‌ها را با ساخت گراف نزدیک‌ترین همسایه در نظر می‌گیرد. بطوریکه، به هر یک از نمونه‌ها بر اساس تعداد همسایه‌های نزدیک به آن وزن نسبت داده می‌شود و همچنین نمونه‌های حاشیه در ایجاد دو ابرصفحه غیر موازی اهمیت ویژه‌ای خواهند داشت. در حالی‌که در روش \lr{TSVM} اصلی، تمام نمونه‌ها در ایجاد ابرصفحه غیرموازی نقش دارند.
%
%در سال 2013، ماشین بردار پشتیبان دو قلو ساختاری\footnote{\lr{Structural Twin Support Vector Machine (STSVM)}}  (\lr{STSVM}) ارائه گردید \cite{qi2013}.  این روش یادگیری اطلاعات مفید ساختاری درون هر کلاس و توزیع نمونه‌ها را از طریق خوشه‌بندی سلسله مراتبی در مدل خروجی لحاظ می‌کند.  در سال 2015، ماشین بردار پشتیبان دو قلو ساختاری با رویکرد گراف نزدیک‌ترین همسایه (\lr{KNN-STSVM}) معرفی شد \cite{pan2015}. در این روش، علاوه بر در نظر گرفتن اطلاعات ساختاری نمونه‌ها، با استفاده از گراف نزدیک‌ترین همسایه به نمونه‌ها وزن داده می‌شود. در نتیجه، دقت دسته‌بند مدل خروجی افزایش می‌یابد.

\section{تعریف مسئله} \label{sec:1:2}
اگرچه ماشین بردار پشتیبان دو قلو نسبت به \lr{SVM} اصلی سریعتر است و داده‌های نامتوزان را بهتر دسته‌بندی می‌کند. با این حال، این روش یادگیری نقاظ ضعفی نیز دارد که عبارتند از:
\begin{enumerate}
	\item 	در این روش، دو مسئله بهنیه‌سازی از نوع برنامه‌ریزی درجه دو باید حل گردد. چناچه نمونه‌های آموزشی بسیار زیاد باشد، سرعت یادگیری این روش به شدت کند می‌شود. زیرا مرتبه زمانی حل کردن یک مسئله بهینه‌سازی درجه دو برابر با $\mathcal{O}(n^3)$ است. بطوریکه $n$ نشان دهنده تعداد نمونه‌های آموزشی می‌باشد. برای رفع کردن این مشکل، از روش \gls{LS}  \cite{kumar2009} استفاده می‌گردد. در نتیجه دو دستگاه معادلات خطی به جای دو مسئله بهینه‌سازی درجه دو حل می‌شود. بنابراین سرعت آموزش دسته‌بند بر روی مجموعه داده‌های بزرگ به طور قابل توجه‌ای افزایش می‌یابد.
	\item برخلاف روش \lr{SVM} اصلی، ماشین بردار پشتیبان دو قلو \gls{ERisk} را در مسئله بهینه‌سازی خود کمینه می‌کند .\cite{shao2011} این مسئله، موجب پدیده  \gls{OF}می‌شود. به عبارت دیگر، مدل خروجی تمام نمونه‌های آموزشی را به خوبی دسته‌بندی می‌کند. بطوریکه دقت مدل خروجی روی نمونه‌های تست کاهش می‌یابد. این مشکل، قدرت  \gls{Gen}روش ماشین بردار پشتیبان دو قلو را کم می‌کند. برای برطرف کردن این مشکل، در سال 2011، شائو و همکاران، ماشین بردار پشتیبان دو قلو مبتنی بر مرز  (\gls{TBSVM}) را ارائه کردند \cite{shao2011}. این روش، به مسئله بهینه‌سازی روش \lr{TSVM} اصلی، یک جمله  \gls{Reg} اضافه می‌کند تا مانند روش \lr{SVM} اصلی حاشیه بیشینه گردد.
	\item ماشین بردار پشتیبان دو قلو به تمام نمونه‌های آموزشی اهمیت یکسانی می‌دهد. در نتیجه ابرصفحه غیر موازی به نمونه‌های نویزی و \gls{Out} نیز نزدیک می‌شود. بنابراین دقت و تعمیم‌پذیری مدل ایجاد شده روی نمونه‌های تست کاهش می‌یابد. ایده اصلی این تحقیق، حل کردن این مسئله است. در پژوهش‌های پیشین نیز به این مسئله پرداخته شده است. برای مثال، روش ماشین بردار پشتیبان دو قلو وزن دار با اطلاعات محلی (\gls{WLTSVM}) \cite{ye2012} با ایجاد گراف نزدیک‌ترین همسایه، به نمونه‌های آموزشی وزن نسبت می‌دهد تا اثر نویز و نمونه‌های پرت در ایجاد مدل برای دسته‌بندی کاهش یابد.
\end{enumerate}

\section{نوآوری‌های پژوهش} \label{sec:1:3}
این پژوهش با ایده گرفتن از روش یادگیری \lr{WLTSVM}  \cite{ye2012}، دو دسته‌بند جدید ارائه می‌کند. بطوریکه نقاط ضعف بیان شده در بخش تعریف مسئله را حل می‌کند. دستاورد این پژوهش شامل معرفی دسته‌بند ماشین بردار پیشتیبان دو قلو کمترین مربعات مبتنی بر نزدیک‌ترین همسایه \cite{mir2018} (\gls{KNN-LSTSVM}) که مزیت‌های زیر را دارد:


\begin{enumerate}
	\item به منظور کاهش اثر نمونه‌های نویزی و پرت، دسته‌بند پیشنهادی \gls*{KNN-LSTSVM} همانند روش \lr{WLTSVM} از گراف نزدیک‌‌ترین همسایه بهره می‌گیرد. بطور‌یکه در مسئله بهینه‌سازی به نمونه‌های آموزشی وزن داده می‌شود و نمونه‌های حاشیه‌ای کلاس مقابل مشخص می‌‌گردد. این ویژگی باعث بهبود دقت دسته‌بندی مدل خروجی می‌شود.
	
	\item دسته‌بند پیشنهادی همانند روش \lr{WLTSVM} باید گراف نزدیک‌ترین همسایه را محاسبه کند تا بتواند وزن تمام نمونه‌های آموزشی را بدست آورد. مرتبه زمانی ایجاد این گراف و ماتریس وزن برابر با $\mathcal{O}(n^{2}logn)$ است. جهت بهبود سرعت یادگیری، دو دستگاه معادلات خطی به جای دو مسئله بهینه‌سازی درجه دو حل می‌شود. از این رو سرعت آموزش روش پیشنهادی به طور قابل توجه‌ای بیشتر از روش \lr{WLTSVM} است و پیاده‌سازی آن نیز ساده‌تر می‌باشد. زیرا دسته‌بند \lr{KNN-LSTSVM} با استفاده از روش کمترین مربعات، نیازی به الگوریتم‌های حل مسائل بهینه‌سازی ندارد.
	
	\item با وزن دهی به نمونه‌ها و در نظرگرفتن نمونه‌های حاشیه‌ای، مدل خروجی در دسته‌بند پیشنهادی نسبت به نمونه‌های نویزی و پرت حساسیت کمتری دارد. زیرا این نمونه‌ها وزن بسیار کمتری دارند.
	
\end{enumerate}

در ادامه با هدف بهبود روش \lr{WLTSVM}، دسته‌بند ماشین بردار پشتیبان دو قلو مبتنی بر رگولارسیون و نزدیک‌ترین همسایه  (\gls{RKNN-TSVM}) ارائه شده است که ويژگی‌های زیر را دارد:

\begin{enumerate}
	\item با وجود اینکه روش \lr{WLTSVM} با استفاده از گراف نزدیک‌ترین همسایه، به نمونه‌های آموزشی وزن می‌دهد، روش وزن‌دهی صرفا بر اساس شمارش تعداد همسایه‌های یک نمونه از طریق گراف نزدیک‌ترین همسایه است.  جهت بهبود شیوه وزن‌دهی به نمونه‌ها، دسته‌بند  (\gls*{RKNN-TSVM}) به یک نمونه آموزشی بر اساس فاصله آن نمونه با همسایه‌های نزدیک خود وزن می‌دهد. مزیت این روش وزن‌دهی این است که نمونه‌های با چگالی بالا\LTRfootnote{High-density samples} بهتر از نمونه‌های نویزی و پرت تفکیک و شناسایی می‌شود. همچنین به نمونه‌هایی که همسایه‌شان نزدیک‌تر است، وزن بیشتری نسبت داده می‌شود.
	
	\item دسته‌بند پیشنهادی (\lr{RKNN-TSVM}) برخلاف روش \lr{WLTSVM} و \lr{TSVM} اصلی، ریسک ساختاری را کمینه می‌کند. بدین منظور یک جمله رگولارسیون به مسئله بهینه‌سازی روش پیشنهادی اضافه شده است. هدف مانند \lr{SVM} اصلی، بیشینه کردن مرز یا حاشیه است. در مجموع، مدل خروجی دچار پدیده برازش بیش از حد نمی‌شود و از تعمیم‌پذیری بهتری برخوردار است.
	
	\item روش \lr{WLTSVM} برای ساخت گراف نزدیک‌ترین همسایه از الگوریتم جستجوی کامل  (\gls{FSA}) استفاده می‌کند که مرتبه زمانی آن برابر با $\mathcal{O}(n^{2})$ است. با این حال روش پیشنهادی (\lr{RKNN-TSVM}) از الگوریتم نزدیک‌ترین همسایه مبتنی بر تفاوت مکانی فاصله‌ها (\gls{LDMDBA}) بهره می‌برد \cite{xia2015}. مرتبه زمانی این الگوریتم برابر با$\mathcal{O}(log d n log n)$  می‌باشد که از الگوریتم \lr{FSA} کمتر است. همچنین الگوریتم \lr{LDMDBA} برای نسخه غیرخطی روش پیشنهادی موثرتر است. زیرا پیدا کردن نزدیک‌ترین همسایه‌های یک نمونه در فضای ویژگی با ابعاد بالا با این الگوریتم سریع‌تر از روش \lr{FSA} می‌باشد.
	
\end{enumerate}

\section{ساختار کلی  پایان‌نامه}\label{sec:1:4}
در فصل \ref{ch:2}، ماشین بردار پشتیبان و گسترش‌های آن از جمله روش \lr{TSVM} بررسی و تشریح شده است. همچنین در این فصل پیشینه پژوهش روش \lr{SVM}  مرور شده است و گسترش‌های این روش نیز معرفی شده‌اند.

دسته‌بند پیشنهادی \lr{KNN-LSTSVM} در فصل \ref{ch:3} ارائه شده است.  نسخه خطی و غیر خطی این دسته‌بند در این فصل تشریح شده است. روش پیشنهادی علاوه بر داشتن مزایای روش  \lr{WLTSVM} ، سرعت یادگیری آن نیز به‌وسیله روش کمترین مربعات بهبود یافته است.

در فصل \ref{ch:4}، دسته‌بند \lr{RKNN-TSVM} معرفی شده است و در ادامه نسخه خطی و غیرخطی آن بیان شده است. این دسته‌بند نقاط ضعف روش \lr{WLTSVM} را حل می‌کند. بطوریکه مدل خروجی با روش جدید وزن‌دهی حساسیت کمتری نسبت به داده‌های نویزی و پرت خواهد داشت.

دو دسته‌بند پیشنهادی \lr{KNN-LSTSVM} و \lr{RKNN-TSVM} در فصل \ref{ch:5} به طور جامع مورد بررسی و ارزیابی قرار گرفته‌اند.   سنجش عملکرد دسته‌بندها با مجموعه داده‌های مصنوعی و واقعی انجام شده است. همچنین سرعت یادگیری نیز با مجموعه داده‌های بزرگ بررسی شده است.

در فصل آخر، ویژگی‌های دسته‌بندهای پیشنهادی مرور شده است. همچنین یافته‌های اصلی این پژوهش نیز در این فصل ذکر شده است. در نهایت پیشنهادهایی برای پژوهش‌های آینده ارائه شده است. 
\newpage
