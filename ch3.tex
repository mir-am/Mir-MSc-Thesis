% Chapter 3 KNN-LSTSVM

\chapter{ماشین بردار پیشتیبان دو قلو کمترین مربعات مبتنی بر نزدیک‌ترین همسایه }\label{ch:3}
\section{مقدمه}\label{sec:3:1}
نقطه ضعف بزرگ روش \lr{TSVM} و \lr{LS-TSVM} این است که این روش‌ها اطلاعات شباهت  بین نمونه‌های آموزشی را در نظر نمی‌گیرند. به عبارت دیگر، به تمام نمونه‌های آموزشی اهمیت یکسانی داده می‌شود. بطوریکه نمونه‌های نویزی و پرت دقت مدل خروجی را روی داده‌های جدید کاهش می‌دهد. روش \lr{WLTSVM} این نقطه ضعف مهم را حل کرده است. این روش با ساخت گراف نزدیک‌ترین همسایه، اطلاعات درون و برون کلاسی را در تابع هدف مسئله بهینه‌سازی لحاظ کرده است. بطوریکه به هر یک از نمونه‌های آموزشی وزن نسبت می‌دهد و همچنین نمونه‌های حاشیه‌ای هر کلاس را استخراج می‌کند.